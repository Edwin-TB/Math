\documentclass{article}
\usepackage[utf8]{inputenc}
\usepackage{amsmath}
\usepackage{geometry}[legalpaper, portrait, margin=1in]

\title{Explaining my Fibonacci Number Generators}
\author{Edwin Trejo}
\date{June 2022}

\begin{document}

\maketitle

\section{Introduction}

    This document aims to explain, in detail, the two basic fibonacci number generators I have written in python.
I am writing this for 1\()\) posterity and 2\()\) I have no other way of applying the Feynman technique at the moment. 
This document also assumes knowledge on what the fibonacci sequence is and how it is generated. I hope you enjoy!

\section{The Recursive Method}

The first generator I made generated the fibonacci numbers recursively as many times as the user wanted.
It begins with a basic set up of the first two terms of the fibonacci sequence:
\begin{verbatim*}
x = 1
y = 1
seq = [x, y]
\end{verbatim*}

The setup is followed by the next_term\(()\) function which generates the next term, and appends it. 

\begin{verbatim*}
    def next_term():
    z = seq[-2] + seq[-1]
    seq.append(z)
\end{verbatim*}

\end{document}