\documentclass{article}
\usepackage[utf8]{inputenc}
\usepackage[legalpaper, portrait, margin=1in]{geometry}
\usepackage{amsmath}

\title{IB Math AA SL Investigation:\emph{Volumes of Cones}}
\author{Edwin Trejo}
\date{April 1, 2022}

\begin{document}

\maketitle

The following gives the page(s) on which questions are answered: \newline
Question 1 - Page 2 \newline
Question 2 - Page 3 \newline
Question 3 - Page 4

\newpage

\begin{enumerate}

    \item Quesiton 1
    \begin{enumerate}
        \item Using data from the table, the system of equations is:
        \begin{align*}
            y(x) = a_0 + a_1x + a_2x^2 + a_3x^3 \\
            y(-1) = a_0 + a_1(-1) + a_2(-1)^2 + a_3(-1)^3 = 4 \\
            y(0) = a_0 + a_1(0) + a_2(0)^2 + a_3(0)^3 = 2 \\
            y(1) = a_0 + a_1(1) + a_2(1)^2 + a_3(1)^3 = 4 \\
            y(2) = a_0 + a_1(2) + a_2(2)^2 + a_3(2)^3 = 4
        \end{align*}
        Simplifying gives:
        \begin{align*}
            a_0 - a_1 + a_2 - a_3 = 4 \\
            a_0 + 0a_1 + 0a_2 + 0a_3 = 2 \\
            a_0 + a_1 + a_2 + a_3 = 4 \\
            a_0 + 2a_1 + 4a_2 + 8a_3 = 4
        \end{align*}
        Constructing an augmented matrix from this system gives:
        $$\begin{bmatrix}
            1 & -1 & 1 & -1 & 4 \\
            1 & 0 & 0 & 0 & 2 \\
            1 & 1 & 1 & 1 & 4 \\
            1 & 2 & 4 & 8 & 4 
        \end{bmatrix}$$
        Where the right-most column is the column containing constant terms \newline
        \item Applying: row operations to the matrix from (a) gives:
        $$\begin{bmatrix}
            1 & 0 & 0 & 0 & 2 \\
            1 & 1 & 1 & 1 & 4 \\
            1 & -1 & 1 & -1 & 4 \\
            1 & 2 & 4 & 8 & 4 
        \end{bmatrix}$$
        $$\begin{bmatrix}
            1 & 0 & 0 & 0 & 2 \\
            0 & 1 & 1 & 1 & 2 \\
            0 & -1 & 1 & -1 & 2 \\
            0 & 2 & 4 & 8 & 2
        \end{bmatrix}$$
        $$\begin{bmatrix}
            1 & 0 & 0 & 0 & 2 \\
            0 & 1 & 1 & 1 & 2 \\
            0 & 0 & 2 & 0 & 4 \\
            0 & 0 & 2 & 6 & -2
        \end{bmatrix}$$
        $$\begin{bmatrix}
            1 & 0 & 0 & 0 & 2 \\
            0 & 1 & 1 & 1 & 2 \\
            0 & 0 & 2 & 0 & 4 \\
            0 & 0 & 0 & 6 & -6
        \end{bmatrix}$$
        $$\begin{bmatrix}
            1 & 0 & 0 & 0 & 2 \\
            0 & 1 & 1 & 1 & 2 \\
            0 & 0 & 1 & 0 & 2 \\
            0 & 0 & 0 & 1 & -1
        \end{bmatrix}$$
        $$\begin{bmatrix}
            1 & 0 & 0 & 0 & 2 \\
            0 & 1 & 0 & 0 & 1 \\
            0 & 0 & 1 & 0 & 2 \\
            0 & 0 & 0 & 1 & -1
        \end{bmatrix}$$
        Which is in RREF.
        \item The RREF matrix from part (b), gives the values of the coefficients of the polynomial to be:
        \[a_0 = 2, a_1 = 1, a_2 = 2, a_3 = -1\]
        Which gives: 
        \[y(x) = 2 + x + 2x^2 - x^3\]
    \end{enumerate}

    \newpage 
    \item Question 2
    \begin{enumerate}
        \item Filling in the places for variables with 0 coefficients in the system gives:
        \begin{align*}
            2x_1 + x_2 + x_3 + 0x_4 + 7x_5 = 20 \\
            x_1 + 0x_2 + x_3 + x_4 + 3x_5 = 9 \\
            0x_1 + x_2 + x_3 + 0x_4 + 5x_5 = 10
        \end{align*}
        Converting this system to an augmented matrix gives:
        $$\begin{bmatrix}
            2 & 1 & 1 & 0 & 7 & 20 \\
            1 & 0 & 1 & 0 & 3 & 9 \\
            0 & 1 & 1 & 0 & 5 & 10
        \end{bmatrix}$$
        \item Applying row operations to the matrix in (a) gives:
        $$\begin{bmatrix}
            1 & 0 & 1 & 0 & 3 & 9 \\
            0 & 1 & 1 & 0 & 5 & 10 \\
            2 & 1 & 1 & 0 & 7 & 20
        \end{bmatrix}$$
        $$\begin{bmatrix}
            1 & 0 & 1 & 0 & 3 & 9 \\
            0 & 1 & 1 & 0 & 5 & 10 \\
            0 & 1 & -1 & 0 & 1 & 2
        \end{bmatrix}$$
        $$\begin{bmatrix}
            1 & 0 & 1 & 0 & 3 & 9 \\
            0 & 1 & 1 & 0 & 5 & 10 \\
            0 & 0 & -2 & 0 & -4 & -8
        \end{bmatrix}$$
        $$\begin{bmatrix}
            1 & 0 & 1 & 0 & 3 & 9 \\
            0 & 1 & 1 & 0 & 5 & 10 \\
            0 & 0 & 1 & 0 & 2 & 4
        \end{bmatrix}$$
        $$\begin{bmatrix}
            1 & 0 & 0 & 0 & 1 & 5 \\
            0 & 1 & 0 & 0 & 3 & 6 \\
            0 & 0 & 1 & 0 & 2 & 4
        \end{bmatrix}$$
        Which is in RREF. 
        \item Translating the augmented matrix in RREF from (b) back into a system of equations gives:
        \begin{align*}
            x_1 + x_5 = 5 \\ 
            x_2 + 3x_5 = 6 \\ 
            x_3 + 2x_5 = 4
        \end{align*}
        Solving for our basic variables, \(x_1\), \(x_2\), and \(x_3\), gives:
        \begin{align*}
            x_1 = 5 - x_5 \\ 
            x_2 = 6 - 3x_5 \\ 
            x_3 = 4 - 2x_5
        \end{align*}
        Thus, our solution set becomes:
        \[ \begin{bmatrix} x_1 \\ x_2 \\ x_3 \\ x_4 \\ x_5 \end{bmatrix} = 
        \begin{bmatrix} 5 - x_5 \\ 6 - 3x_5 \\ 4 - 2x_5 \\ x_4 \\ x_5 \end{bmatrix} = 
        \begin{bmatrix} 5 \\ 6 \\ 4 \\ 0 \\ 0  \end{bmatrix} + 
        \begin{bmatrix} 0 \\ 0 \\ 0 \\ x_4 \\ 0 \end{bmatrix} + 
        \begin{bmatrix} - x_5 \\ - 3x_5 \\ - 2x_5 \\ 0 \\ x_5 \end{bmatrix} = \]
        \[ \begin{bmatrix} 5 \\ 6 \\ 4 \\ 0 \\ 0  \end{bmatrix} + 
        x_4 \begin{bmatrix} 0 \\ 0 \\ 0 \\ 1 \\ 0 \end{bmatrix} + 
        x_5 \begin{bmatrix} -1 \\ -3\\ -2 \\ 0 \\ 0 \end{bmatrix}
        \]
    \end{enumerate}

    \newpage 
    \item Question 3
    \begin{enumerate}
        \item Name: Edwin Trejo Balderas; Facilitator: LaRosa Johnson; School: Douglas County High School
    \end{enumerate}

\end{enumerate}


\end{document}