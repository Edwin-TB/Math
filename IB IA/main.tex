\documentclass[11pt]{article}
\usepackage[utf8]{inputenc}
\usepackage[legalpaper, portrait, margin=1in]{geometry}
\usepackage{amsmath}
\usepackage{pgfplots}
\usepackage{setspace}
\usepackage{indentfirst}

\newtheorem{theorem}{Theorem}[section]
\newtheorem{lemma}[theorem]{Lemma}
\newtheorem{corollary}[theorem]{Corollary}
\newtheorem{definition}[theorem]{Definition}

\usepgfplotslibrary{external}
\tikzexternalize
\pgfplotsset{width=8cm,compat=1.9}
\setlength{\parindent}{20pt}


\title{IB Math Analysis and Approaches Internal Assessment: \emph{Generalizing and Describing the Fibonacci Sequence}}
\author{Edwin Trejo Balderas}
\date{December, 2022}

\begin{document}

\singlespacing

\maketitle

\doublespacing

\section{Introduction and Rationale}

I have always been fascinated by identifying and describing patterns, especially those that appear in numbers. I constantly observe these patterns and attempt to describe a relationship between terms so that I can know exactly what to expect next. One object that especially lights this fire is the fibonacci sequence, which always seemed so simple in construction but elusive in significance. I never knew how it connected to its visualizations such as the golden spiral, golden angle, or buildings made with golden proportions. Nor did I know of any methods to explicitly find an arbitrarily-high index in the sequence, if any. I also have an interest in generalizations or extensions of any mathematical concept or method as I do not like being constrained and like observing the new properties that arise. I wondered what factorials for non-integers would look like, so I defined my own version before I discovered the gamma function. Similarly, I now wonder what the fibonacci sequence would behave like if every subsequent term was the sum of the previous three or more terms instead of the sum of the previous two. 

This past summer, my mathematical abilities developed greatly and I learned how to treat the subject almost as a form of art, “playing” with new concepts I learned in an attempt to extract a new observation or property. This has allowed me to better understand what intuitive extensions of mathematical concepts would look like and how to deal with their consequences or results more naturally. I also learned about constant recursive sequences, how the fibonacci sequence is just one kind of these, their relationship to homogeneous differential equations, and methods of determining explicit formulas for second-degree recursive relationships. I hope to further develop my mathematical maturity so that in the future I can make significant contributions to the field, and I believe this investigation is a good place to start.

\section{Aim and Methodology}

In this investigation, I aim to determine the behavior of generalized n-nacci sequences by finding the ratios they approach, which I will call their ‘limiting ratio’ as the index approaches infinity and finding explicit formulas for their recursive relationships. I will first investigate these properties of the fibonacci sequence, using algebra to show that their limiting ratio is indeed the golden ratio and methods similar to those used to find solutions of homogeneous differential equations to find binet’s formula. I will then define n-nacci sequences and try to extend methods to these new sequences. I will use computer code to simplify the process of finding explicit formulas and generating high indexes of these formulas to postulate their limiting ratios. I will then use geometric sum formulas to prove these ratios, and derive the ratio that appears when infinitely many previous terms are added. Ultimately, I will use my observations to determine a relationship between finite and infinite geometric sums when the final sum is held constant but the amount of terms added varies. 

\noindent Commonly used symbols are summarized in the table below:

\begin{table}[h]
\begin{center}
    \begin{tabular}{ |c|c| } 
    \hline
    \textbf{Symbol} & \textbf{Value} \\
    \hline
    $a_n$ & The n-th term in a sequence \\ 
    \hline
    $r$ & The ratio between the n-th term in a sequence and the (n-1)-th term \\ 
    \hline
    $F_n$ & The n-th term in the fibonacci sequence \\
    \hline
    $T_n$ & The n-th term in the tribonacci sequence \\
    \hline
    $F^{(m)}_n$ & The n-th term in the m-acci sequence \\ 
    \hline
    $\Phi_m$ & The limiting ratio of the m-nacci sequence \\
    \hline
    \end{tabular}
    \caption{Commonly Used Symbols.}
\label{table:1}
\end{center}
\end{table}

\section{Linear Homogenous Recurrence Relations}
A few fundamental definitions are given below. The first two come from Bozlee (2017), and the third from a course in sequences and series I took over the summer before my senior year.

\begin{definition}[Linear Homogenous Recurrence Relation] A linear homogenous recurrence relation (also called a linear recurrence) of degree \mbox{k}, is a recurrence relation of the form:
    $$a_n=c_1a_{n-1}+c_2a_{n-2}+\dots +c_ka_{n-k}; c_1,c_2,\dots,c_k\in \mbox{\textbf{C}}$$ 
    With initial terms: 
    $$a_1,a_2,\dots,a_k\in\mbox{\textbf{C}}$$
\end{definition}

\begin{definition}[Characteristic Polynomial] The characteristic polynomial of a linear recurrence is a polynomial of the form:
    $$f(\lambda) = \lambda^{k}-c_1\lambda^{k-1}-c_2\lambda^{k-2}-\dots-c_{k-1}\lambda-c_k$$ 
\end{definition}

\begin{definition}[Explicit Form] The explicit form of a linear recurrence is an expression of the form:
    $$a_n=d_1\lambda_1^n+d_2\lambda_2^n+\dots+d_k\lambda_k^n; d_1,d_2,\dots,d_k \in\mbox{\textbf{C}}$$ 
    Where \mbox{$\lambda_1,\lambda_2,\dots,\lambda_k$} are the roots of the characteristic polynomial
\end{definition}

\doublespacing
\subsection{Worked Example}
As an example, we can determine the explicit form of the following linear recurrence:
$$a_n=4a_{n-1}-3a_{n-2}$$
Which has the following characteristic polynomial:
$$\lambda^2-4\lambda+3$$

\end{document}