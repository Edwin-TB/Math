\documentclass[12pt, letterpaper]{article}
\usepackage[utf8]{inputenc}
\usepackage{tikz}
\usepackage[legalpaper, portrait, margin=0.5in]{geometry}
\usepackage{amsthm}
\setlength{\parindent}{0pt}
\newcommand{\forceindent}{\leavevmode{\parindent=1em\indent}}

\theoremstyle{definition}
\newtheorem{definition}{Definition}

\theoremstyle{definition}
\newtheorem{exmp}{Example}


\title{An Expression on the Beauty of Mathematics}
\author{Edwin Trejo Balderas}
\date{August 2022}

\begin{document}

\maketitle

\section{Introduction}
\forceindent This document aims to offer a glimpse into the independent math work I do in my free time as unfortunately, I do not believe the Activities section on the WISE application offers me enough space to do that.
In this document, I show a new method of deriving the formula for the polygonal numbers using mathematical concepts I learned at the Georgia Governor's Honors Program. Special thanks to Vinton Geistfeld for the excellent instruction throughout the program.


\section{Background}
\forceindent The polygonal numbers are a set of integer sequences that represent the number of dots required to make the shape of a regular polygon of a given size. 
Each polygonal number has 2 parameters, the number of sides, \(n\), and the side length, \(s\), of the shape it creates. These are combined to give \(P(n,s)\).


\begin{exmp}
    The polygonal number \(P(3,2)\) represents the number of dots required to make a triangle of side length 2, shown below:
    \begin{figure}[h]
        \centering
        \begin{tikzpicture}
        \filldraw[black] (0,0) circle (2pt);
        \filldraw[black] (-0.7,-1) circle (2pt);
        \filldraw[black] (0.7,-1) circle (2pt);
        \end{tikzpicture} 
        \label{A Visual Representation of the Polygonal Number P(3,2)}
    \end{figure}
\newline Thus, \(P(3,2) = 3\).
\end{exmp}

\forceindent Another important concept is the \textbf{finite difference}, akin to the derivative of a continuous function, it gives the difference between two successive terms in a sequence.

\begin{definition}
The finite difference for a sequence, \(f(x)\) is deinfed as:
\[\Delta[f(x)] = f(x+1) - f(x)\]
\end{definition}

\forceindent The reverse of the finite difference is the finite sum, which is analagous to the integral of a 

Applying the finite difference operator to the first 5 triangular numbers gives:
\[f(x) = 1,3,6,10,15\]
\[\Delta[f(x)] = (3-1), (6-3), (10-6), (15-10) = 2,3,4,5\]
Applying it once more gives: 
\[\Delta^2[f(x)] = (3-2),(4-3),(5-4) = 1,1,1\]
From this, we can derive a formula 

\end{document}