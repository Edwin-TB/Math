\documentclass[11pt]{article}
\usepackage[utf8]{inputenc}
\usepackage[legalpaper, portrait, margin=1in]{geometry}
\usepackage{amsmath}
\usepackage{pgfplots}
\usepackage{setspace}
\usepackage{indentfirst}

\newtheorem{theorem}{Theorem}[section]
\newtheorem{lemma}[theorem]{Lemma}
\newtheorem{corollary}[theorem]{Corollary}
\newtheorem{definition}[theorem]{Definition}
\newtheorem{proposition}[theorem]{Proposition}

\usepgfplotslibrary{external}
\tikzexternalize
\pgfplotsset{width=8cm,compat=1.9}
\setlength{\parindent}{20pt}

\title{Two Quick Proofs as an Expression of Affection for Mathematics}
\author{Edwin Trejo Balderas}
\date{August 2022}

\begin{document}

\maketitle

\section{Introduction}
This document aims to offer a glimpse into the independent math work I do in my free time as, unfortunately, I do not believe the activities section on the WISE application offers me enough space to do that. 
I also hope it serves as sufficient proof of my deep love for mathematics and interest in MIT's course 18.
In this document, I prove that every positive even power of 2 is one more than a multiple of 3 and every positive odd power of 2 is one less than a multiple of 3 using mathematical induction. 
I came up with the propositions presented by accident, simply wondering about the powers of 2 and the numbers near them, which is how many of my propositions originate. This creativity that math allows is what I mean when I say it is an art. 

\section{Observations and Propositions}
The first few powers of 2 (starting at \(n=1\)) are shown below:
\[a_n = 2^n = 2,4,8,16,32,64,\dots\]
Notice how adding 1 to each of 2, 8, and 32 gives a multiple of 3:
\[2 + 1 = 3 = 3(1)\]
\[8 + 1 = 9 = 3(3)\]
\[32 + 1 = 33 = 3(11)\]
Also notice how 2, 8, and 32 are odd powers of 2:
\[2 = 2^1, 8=2^3, 32=2^5\]

From this, I propose this holds for all odd powers of 2:

\begin{proposition}
    \[2^{2n-1} + 1 = 3(x) \mbox{ such that x is an integer for all } n = 1,2,3,\dots\]
\end{proposition}

Alternatively, we notice how subtracting 1 from each of 4, 16, and 64 gives a multiple of 3:
\[4 - 1 = 3 = 3(1)\]
\[16 - 1 = 15 = 3(5)\]
\[64 - 1 = 63 = 3(21)\]
Also notice how 4, 16, and 64 are even powers of 2:
\[4 = 2^2, 16=2^4, 64=2^6\]

From this, I propose this holds for all odd powers of 2:

\begin{proposition}
    \[2^{2n} - 1 = 3(x) \mbox{ such that x is an integer for all } n = 1,2,3,\dots\]
\end{proposition}

\section{Proof of Proposition 1}
To prove the following:
\[2^{2n-1} + 1 = 3(x) \mbox{ such that x is an integer for all } n = 1,2,3,\dots\]
We first must show it is true for the case where \(n=1\), indeed:
\[2^{2(1)-1} + 1 = 2 + 1 = 3\]
Now, assuming \(n=k\) for all \(k = 1,2,3,\dots\), we have:
\[2^{2k-1} + 1 = 3(x) \mbox{ such that x is an integer for all } k = 1,2,3,\dots\]
It follows that this is also true for all \(n=k+1\):
\[2^{2(k+1)-1} + 1 = 3(x)\]
\[2^{2(k+1)-1} + 1 = 2^{2k-1 + 2} + 1 = 2^2 2^{2k-1} + 1 = 4(2^{2k-1}) + 1 = 3(2^{2k-1}) + 2^{2k-1} + 1 \]
From our assumption, we know that \(2^{2k-1} + 1\) is in fact a multiple of 3, so we can replace it with \(3m\) where \(m\) is some integer:
\[3(2^{2k-1}) + 3m = 3(2^{2k-1} + m)\]
Since both factors of 3 are integers, we have shown that the above expression is indeed a multiple of 3, and have proven proposition 1

\section{Proof of Proposition 2}
The proof of the second proposition, shown below, follows a very similar progression as the first.
\[2^{2n} - 1 = 3(x) \mbox{ such that x is an integer for all } n = 1,2,3,\dots\]

For \(n=1\), indeed:
\[2^2 - 1 = 4-1 = 3\]
Assuming \(n=k\) for all \(k = 1,2,3,\dots\), we have:
\[2^{2k} - 1 = 3(x) \mbox{ such that x is an integer for all } k = 1,2,3,\dots\]
It follows that this is also true for all \(n=k+1\):
\[2^{2(k+1)} - 1 = 3(x)\]
\[2^{2(k+1)} - 1 = 2^2 2^k -1 = 4(2^k)-1= 3(2^k) + 2^k-1 = 3(2^k) + 3m = 3(2^k+m)\]
Thus proving proposition 2.

\section{Bonus Proof}
Since all even powers of 2, \(2^{2n}\), can be simplified to powers of 4, \(4^n\), we can see that, using the proof of proposition 2, the following holds:
\[4^{n} - 1 = 3(x) \mbox{ such that x is an integer for all } n = 1,2,3,\dots\]

\section{Extensions}
Currently, I am working towards generalizing these two theorems to the rest of the integers, such as even/odd powers of 3 being 1 greater/less than multiples of 4, 
or rather even/odd powers of \(n\) being 1 greater/less than multiples of \(n+1\). I am also working to see if this holds in reverse, where all even/odd powers of \(n\) could be 1 greater/less than multiples of \(n-1\).
Thank you so much for reading! :)

\end{document}